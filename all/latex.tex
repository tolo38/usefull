\documentclass{article}
	\title{Creer avec \LaTeX}
	\usepackage{amssymb}
\begin{document}
	\maketitle
	\newpage
	\tableofcontents
	\newpage
	\section*{Type de document}
		$\backslash documentclass\{minimal\} $
		- même pas de titire\\ %\leftarrow
		$\backslash documentclass\{article\}$
		- section et sous section\\
		$\backslash documentclass\{report\} $
		- Partie et chapitre en plus des sections\\
	\section*{Corp du document}
		$\backslash begin\{document\}$ \\
		ecrir ici 
		$\backslash \backslash$ \%Commentaire : le double backslash permet de retourner a la ligne\\
		$\backslash newline$ \\
		$\backslash newpage$ \\
		$\backslash end\{document\}$ \\
	\section*{Organisation}
		pour annoncer une partie\\
		$\backslash part\{Nom De La Partie\}$ \\
		\newline
		pour annoncer un chapitre\\
		$\backslash chapter\{Nom Du Chapitre\}$ \\
		\newline
		pour annoncer une section\\
		$\backslash section\{Nom De La Section\}$ ou  $\backslash section*\{Nom De La Section\}$ si on ne veux pas le numero \\
		\newline
		pour annoncer une sous-section\\
		$\backslash subsection\{Nom De La Sous Section\}$ \\
		\newline
		pour annoncer un paragraphe\\
		$\backslash subsection\{Nom De La Sous Section\}$ \\
		\newline
		pour annoncer un sous-paragraphe\\
		$\backslash subsection\{Nom De La Sous Section\}$ 
	\section*{Un titre}
		Creation du titre \\
		$\backslash title\{Titre Du Document\}$
		entre le document class et le begin \\		
		Position du titre a l'endroit ou l'on peut lire \\
		$\backslash maketitle$ \\
	\section*{Liste}	
	\section*{Sommaire et Index}	
	\section*{Package}	
%		\usepackage[utf8]{inputenc}
%		\usepackage[latin1]{inputenc}
	\section*{Image}	
	\section*{Math}	
	\subsection*{Symbole}	
		\[\sum_{n=1}^{+\infty}\frac{1}{n^2}=\frac{\pi^2}{6}.\]
	\section*{Informatique : ecrire du code source}	
	\section*{Eciture}	
	\subsection*{Allignement}	
	\subsection*{Police}
	\section*{Tableau}	
		\begin{tabbing}
		Une phrase assez longue \= encore rallongée \= qui se finit ici. \= \\
		Fruits  \> pomme \> poire \> prune \\
		Couleurs \> vert \>jaune \> violet
		\end{tabbing}
\end{document}
